The SVS method of learning shape representation, specifically its keypoint search part, was adapted to search keypoints in range image data. In order to do so, a problem of making SVM work subquadratically was solved. However, both quality of the obtained keypoints and computation time were significantly worse than ones of NARF keypoint extraction method. The possible reason could be that Kinect data was used, that is known to be especially noisy. The NARF keypoint extraction algorithm was developed to be robust to noise, whereas SVS-KP method is based only on general considerations of classifying interior and exterior of some shape. However, the fact that performance was almost similar for ``dumb'' version of SVS-KP algorithm testifies, that SVM optimization is not very relevant for the considered problem.

It would be interesting in future work to combine both detailed problem awareness of NARF and elegant idea to use analytical properties of some continuous valued functions for feature point search, as it is done in SVS approach.