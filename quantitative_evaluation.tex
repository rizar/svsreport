In the first experiment I compared stability of keypoints obtained with different values of parameters. In addition in order to understand if running SVM makes sense at all I tried a ``dumb'' version of the algorithm which simply sets all $\alpha_i$ equal to $C$. To show which of the quantities testify the performance better I add the random keypoint selection to the comparison. 30 Kinect scans from the first second of ``fr1/xyz'' sequence served as data. The following quantities were measured:
\begin{itemize}
\item
$K$ ~--- average number of keypoints
\item
$K_i$ ~--- average number of keypoints that moved on more than $i$ centimeters whens current Kinect scan was replaced by the next one, $i=1,\ldots,3$.
\item
$\kappa_i$ ~--- average percentage of keypoints that move on no more than $i$ centimeters, $i=1,\ldots,3$
\end{itemize}

The table \ref{figure:table1} presents the experiment results. One can see that the most successful performance indicator is $\kappa_1$: it clearly separates all the variations of the algorithm from random. Another observation is that large values of $C$ do deteriorate performance as it was predicted in subsection \ref{sec:params}. In fact, the performance was improving all the way the $C$ parameter was increased. For $C=0.125$ I tried to vary other free parameters such as $\alpha_{gap}$ and $K_1$. No profit was obtained in such a way, showing that those picked first were close to the optimal. The shocking result was that  using SVM doesn't matter that much: the ``dumb'' version of algorithm performed very close to the SVM one. For $C=0.125$ the ``dumb'' was even better, however the best performance in terms of $\kappa_{1}$ was still obtained with SVM.

In the second experiment I compared the best of SVM-based algorithms with NARF. The same data was used. The support size of NARF was $0.1$, for other parameters PCL default settings were used. As one can see in figure \ref{figure:beda} the SVS-KP totally lost NARF in terms of keypoint stability. Both in terms of absolute number and percentage of stable keypoints NARF performed significantly better. It was also much faster: $\approx$ 6 versus $\approx$ 170 seconds per frame. Having extreme values of $C=0.005$ caused SVM to use almost all training set examples as support vectors. For more moderate values $C$, e.g. $0.125$, support vectors made only about $10\%$ of all the examples and computation time was about 40 seconds per frame. However, the performance was correspondingly lower. 

\begin{figure}
	\centering
\begin{tabular}{|c|c|c|c|c|c|c|c|}
\hline
& $K$ & $K_1$ & $K_2$ & $K_3$ & $\kappa_1$ & $\kappa_2$ & $\kappa_3$ \\

\hline % 140.par
$C = 16$ & $126.69$ & $12.83$ & $28.00$ & $44.17$ & $11.22$ & $23.56$ & $36.16$ \\

\hline % 130.par
$C = 2$ & $135.79$ & $15.59$ & $35.07$ & $53.55$ & $12.67$ & $27.23$ & $40.81$ \\

\hline % 120.par
$C=0.5$ & $143.62$ & $19.10$ & $39.14$ & $59.86$ & $14.45$ & $28.65$ & $43.06$ \\

\hline % 110.par
$C = 0.125$ & $150.62$ & $20.72$ & $44.00$ & $66.07$ & $14.97$ & $30.64$ & $45.33$ \\

\hline % 111.par
$C = 0.125$, $\alpha_{gap}=0.1$ & $169.97$ & $21.10$ & $48.41$ & $74.21$ & $13.56$ & $29.93$ & $45.25$ \\

\hline % 112.par
$C = 0.125$, $\alpha_{gap}=0.3$ & $146.10$ & $16.79$ & $37.10$ & $59.31$ & $12.33$ & $26.51$ & $41.72$ \\

\hline % 113.par
$C = 0.125$, $K_1=10^{-1}$ & $137.41$ & $17.31$ & $38.17$ & $57.38$ & $13.74$ & $29.28$ & $43.22$  \\

\hline % 114.par
$C = 0.125$, $K_1=10^{-3}$ & $158.76$ & $20.76$ & $43.38$ & $68.62$ & $14.32$ & $28.61$ & $44.44$ \\
	
\hline % 100.par
$C = 0.03$ & $157.28$ & $24.07$ & $46.79$ & $68.48$ & $16.71$ & $31.30$ & $45.14$ \\	

\hline % 90.par
$C = 0.005$ & $138.03$ & $21.28$ & $43.14$ & $61.97$ & $17.09$ & $33.35$ & $46.98$ \\	
	
\hline % 110.par + dumb
$C = 0.0125$, ``dumb'' & $169.41$ & $24.22$ & $48.89$ & $72.59$ & $15.81$ & $30.72$ & $44.68$ \\	

\hline % 90.par + dumb
$C = 0.005$, ``dumb'' & $164.10$ & $23.97$ & $48.10$ & $71.10$ & $16.22$ & $31.36$ & $45.36$ \\	


\hline % random
random & $163.0$ & $8.84$ & $32$ & $57.84$ & $5.4$ & $19.61$ & $35.57$ \\

\hline
\end{tabular}
\caption{Evaluation of keypoints stability for different values of parameters and for ``dumb'' method. Support size $\sigma$ is fixed to be $0.1$.}
\label{figure:table1}
\end{figure}

\begin{figure}
\includegraphics[scale=0.75]{svs_vs_narf.eps}
\caption{SVS vs NARF. The upper plot displays number of keypoints that moved less than on $0.1 cm$ as it changes over time. The lower displays percentage of such keypoints.}
\label{figure:beda}
\end{figure}