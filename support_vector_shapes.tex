\subsection{SVM-based shape representation}
In the classification task one has to reconstruct space separation into two regions using examples from both. In \cite	{SVS2013} in some sense opposite approach is proposed: given a shape (that is separation of the plane into shape interior and exterior) represent it with a decision function. In order to do it the training set is sampled randomly both from interior and exterior of the shape. The points with a high norm of decision function gradient $||\nabla f||$ are used as feature points. As a descriptor the Histogram of Oriented Gradient (HOG) approach was applied, with a notable  distinction from original approach \cite{dalal2005histograms}: instead of the image gradient the decision function gradient $\nabla f$ was used. This scheme complemented with complicated $DP$ matching schemed allowed them obtain close to state-of-the-art performance in the shape retrieval benchmark.

The motivation to use the gradient norm maximums as feature points comes from experiments which showed their robustness and independence from SVM hyperparameters.  I apply the same approach for 3D range images, as they are essentially are a set of 3D shapes. However, the dramatic difference beetween learning 2D and 3D boundary is the number of points they contain and consequently different requirements on the complexity of learning algorithm. The next section is entirely devoted to this problem.
